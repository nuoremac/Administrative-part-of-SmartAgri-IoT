\documentclass[11pt,a4paper]{article}

\usepackage[utf8]{inputenc}
\usepackage[T1]{fontenc}
\usepackage{lmodern}
\usepackage[french]{babel}
\usepackage[a4paper,margin=2.5cm]{geometry}
\usepackage{graphicx}
\usepackage{float}
\usepackage{hyperref}
\usepackage{enumitem}
\usepackage{xcolor}
\usepackage{array}

\hypersetup{
  colorlinks=true,
  linkcolor=blue,
  urlcolor=blue
}

\title{Smart-Agro IoT\\Guide d'utilisation}
\author{Équipe projet Smart-Agro}
\date{\today}

% Mettre à true lorsque les captures sont disponibles.
\newif\ifaffichercaptures
\affichercapturesfalse

% #1: nom du fichier dans screenshots/
% #2: légende
% #3: identifiant du label
\newcommand{\captureguide}[3]{
  \begin{figure}[H]
    \centering
    \ifaffichercaptures
      \includegraphics[width=0.95\textwidth]{screenshots/#1}
    \else
      \fbox{
        \parbox[c][0.26\textheight][c]{0.90\textwidth}{
          \centering
          \textbf{Capture à insérer}\\[4pt]
          Fichier attendu : \texttt{screenshots/#1}
        }
      }
    \fi
    \caption{#2}
    \label{fig:#3}
  \end{figure}
}

\begin{document}

\maketitle
\tableofcontents
\newpage

\section{Objet du document}
Le présent guide a pour objectif d'accompagner l'utilisateur dans la prise en main de la plateforme Smart-Agro IoT. Il décrit, de manière progressive, les opérations courantes réalisées depuis l'interface web : authentification, consultation des données, gestion des entités métier et administration des préférences du compte.

Ce document est destiné à servir de référence opérationnelle. Il peut être utilisé à la fois comme support d'apprentissage initial et comme aide rapide lors de l'exploitation quotidienne de l'application.

\section{Périmètre fonctionnel}
Le guide couvre les modules actuellement disponibles dans le périmètre administrateur : tableau de bord, gestion des utilisateurs, gestion des terrains, parcelles et capteurs, notifications, profil et déconnexion.

Les fonctionnalités très techniques (maintenance système, opérations internes backend, etc.) ne font pas partie du périmètre de ce document.

\section{Public visé}
Cette documentation s'adresse à deux profils principaux :
\begin{itemize}[leftmargin=1.2em]
  \item \textbf{Administrateur} : il supervise les comptes et pilote les opérations de gestion globale.
  \item \textbf{Utilisateur métier} : il exploite les fonctionnalités liées aux terrains, parcelles, capteurs et notifications.
\end{itemize}

\section{Prérequis}
Avant d'utiliser l'application, il est nécessaire de disposer d'un compte actif, d'un navigateur récent et d'un accès réseau au service Smart-Agro.

Pour garantir une utilisation fluide, il est recommandé d'utiliser une connexion stable et de vérifier les droits associés au profil connecté.

\section{Conventions de lecture}
Dans ce guide, les actions utilisateur sont décrites dans l'ordre d'exécution réel. Les libellés de boutons et de menus apparaissent en \texttt{monospace} (exemple : \texttt{Save}, \texttt{Logout}) pour faciliter leur identification dans l'interface.

Les captures d'écran sont référencées dans le dossier \texttt{docs/screenshots/}. Tant que les images finales ne sont pas disponibles, des emplacements sont affichés automatiquement dans le PDF.

\section{Parcours d'utilisation}

\subsection{Connexion}
La connexion constitue le point d'entrée de la plateforme. L'utilisateur ouvre la page d'authentification, saisit son email et son mot de passe, puis valide via le bouton \texttt{Sign in}. Si les informations sont correctes et que le compte est autorisé, la session est ouverte et l'accès au tableau de bord est immédiat.

En cas d'échec, il convient de vérifier les identifiants saisis ainsi que le statut du compte.
\captureguide{01-login-page.png}{Écran de connexion}{connexion}

\subsection{Tableau de bord}
Le tableau de bord fournit une vue synthétique de l'état de la plateforme. Il permet d'identifier rapidement les indicateurs clés et de confirmer que les données principales sont disponibles.

Cette page est généralement la première étape de vérification après la connexion, avant de naviguer vers les modules de gestion détaillée.
\captureguide{02-dashboard.png}{Tableau de bord Smart-Agro}{tableau-de-bord}

\subsection{Gestion des utilisateurs (profil administrateur)}
Le module \texttt{Users} est réservé à l'administration des comptes. L'administrateur peut consulter la liste des utilisateurs, créer de nouveaux comptes, mettre à jour les informations existantes et supprimer ou désactiver les accès selon les besoins.

Après chaque action de création ou de modification, il est recommandé de contrôler la cohérence des informations affichées dans la liste.
\captureguide{03-users-page.png}{Gestion des utilisateurs}{utilisateurs}

\subsection{Gestion des terrains}
La page \texttt{Terrains} centralise les opérations liées aux terrains agricoles. L'utilisateur peut y créer un nouveau terrain, compléter ses métadonnées et accéder à sa fiche de détail.

Cette étape est structurante, car les parcelles sont rattachées aux terrains existants. Une bonne qualité de saisie à ce niveau facilite l'ensemble des traitements aval.
\captureguide{04-terrains-page.png}{Gestion des terrains}{terrains}

\subsection{Gestion des parcelles}
Le module \texttt{Parcels} permet de consulter les parcelles enregistrées et leurs associations métiers. L'utilisateur peut rechercher une parcelle, ouvrir sa fiche et vérifier les informations clés, notamment le terrain parent et la superficie.

La consultation régulière de ce module permet d'assurer l'alignement entre les données terrain et les données de suivi opérationnel.
\captureguide{05-parcels-page.png}{Liste des parcelles}{parcelles}

\subsection{Gestion des capteurs}
La page \texttt{Sensors} est dédiée à l'intégration des capteurs IoT et au suivi de leur état. L'utilisateur peut créer un capteur, consulter ses caractéristiques techniques et valider les identifiants nécessaires à l'exploitation (code, identifiant appareil, état).

Une vérification systématique des informations techniques après création est conseillée pour éviter les erreurs de collecte.
\captureguide{06-sensors-page.png}{Gestion des capteurs}{capteurs}

\subsection{Notifications}
Le module \texttt{Notifications} sert à définir les canaux de notification autorisés et à valider leur fonctionnement via un test d'envoi.

Le flux recommandé consiste à sélectionner les canaux, enregistrer les paramètres, puis exécuter un test afin de confirmer le bon acheminement des messages.
\captureguide{07-notifications-page.png}{Paramétrage des notifications}{notifications}

\subsection{Profil utilisateur}
Depuis le menu \texttt{Profile}, l'utilisateur met à jour ses informations personnelles (identité, téléphone, avatar) et gère la sécurité de son compte via la modification du mot de passe.

Après sauvegarde, les changements doivent être visibles immédiatement dans les zones d'affichage du profil et de la barre latérale.
\captureguide{08-profile-page.png}{Gestion du profil utilisateur}{profil}

\subsection{Langue et thème}
L'application propose des options d'affichage permettant d'adapter l'interface aux préférences de l'utilisateur : choix de la langue et bascule entre les thèmes clair et sombre.

Ces paramètres améliorent le confort d'utilisation sans modifier le contenu métier.
\captureguide{09-theme-language.png}{Réglages de langue et thème}{langue-theme}

\subsection{Déconnexion}
La déconnexion clôture la session de travail. L'action se fait depuis la barre latérale via \texttt{Logout}. Une fois exécutée, l'utilisateur est redirigé vers la page de connexion et la session active est invalidée.

Cette étape doit être systématiquement réalisée en fin d'utilisation, en particulier sur un poste partagé.
\captureguide{10-logout.png}{Déconnexion de l'application}{deconnexion}

\section{Gestion des incidents courants}
\subsection{Échec de chargement des données}
En cas de message d'erreur de chargement, vérifier d'abord la connectivité réseau puis la disponibilité du backend Smart-Agro. Si l'incident persiste, contrôler les droits du compte connecté.

\subsection{Erreur d'autorisation (401/403)}
Ce type d'erreur indique généralement une session expirée ou un niveau de permission insuffisant. Il est recommandé de se déconnecter, puis de se reconnecter avec un compte disposant des droits requis.

\subsection{Échec d'upload d'avatar}
Vérifier le format du fichier image (JPG/PNG/WebP) et réduire sa taille si nécessaire avant de relancer l'opération.

\section{Liste des captures attendues}
\begin{enumerate}[leftmargin=1.4em]
  \item \texttt{screenshots/01-login-page.png}
  \item \texttt{screenshots/02-dashboard.png}
  \item \texttt{screenshots/03-users-page.png}
  \item \texttt{screenshots/04-terrains-page.png}
  \item \texttt{screenshots/05-parcels-page.png}
  \item \texttt{screenshots/06-sensors-page.png}
  \item \texttt{screenshots/07-notifications-page.png}
  \item \texttt{screenshots/08-profile-page.png}
  \item \texttt{screenshots/09-theme-language.png}
  \item \texttt{screenshots/10-logout.png}
\end{enumerate}

\section{Historique des versions}
\begin{center}
\begin{tabular}{|>{\raggedright\arraybackslash}p{2.2cm}|>{\raggedright\arraybackslash}p{2.5cm}|>{\raggedright\arraybackslash}p{3.0cm}|>{\raggedright\arraybackslash}p{5.2cm}|}
\hline
\textbf{Version} & \textbf{Date} & \textbf{Auteur} & \textbf{Modification} \\
\hline
1.1 & \today & Équipe Smart-Agro & Réécriture du guide en style narratif et professionnel. \\
\hline
\end{tabular}
\end{center}

\section{Compilation PDF}
\begin{verbatim}
cd docs
pdflatex guide-utilisateur-smart-agro.tex
pdflatex guide-utilisateur-smart-agro.tex
\end{verbatim}

\end{document}
